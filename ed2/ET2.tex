\documentclass[12pt]{article}
\usepackage[brazil]{babel}
\usepackage[utf8]{inputenc}
\usepackage[T1]{fontenc}
\usepackage{amsmath}
\usepackage{hyperref}
\usepackage{enumitem}
\usepackage{geometry}
\geometry{a4paper, margin=2.5cm}
\usepackage{parskip}
\setlength{\parindent}{0pt}

\title{Estudo Teórico 2 \\ \large ELT091 – Redes TCP/IP – 2025/1}
\author{
Guilherme Astolfo Rigacci \\
Augusto Ribeiro \\
Matheus Miranda
}
\date{24 de abril de 2025}

\begin{document}

\maketitle

\section*{Questão 1}

\textbf{a)} Uma RFC (Request for Comments) é um tipo de publicação formal do IETF (Internet Engineering Task Force), usada para desenvolver e definir padrões da Internet. Elas podem conter especificações técnicas, diretrizes, ou experimentações. O órgão responsável por gerenciar essas RFCs é o próprio IETF, sob supervisão da IAB (Internet Architecture Board). 

A diferença entre uma RFC do tipo \textbf{Standards Track} e uma do tipo \textbf{Experimental} está no propósito: as primeiras seguem um processo formal de padronização e podem se tornar um padrão oficial da Internet, enquanto as experimentais propõem ideias ou tecnologias que ainda não estão maduras o suficiente para padronização. 

\textbf{Fonte:} RFC Editor – \url{https://www.rfc-editor.org/about/}, acesso em abr. 2025.

\vspace{0.5em}
\textbf{b)} A RFC 791 descreve o protocolo IP. O campo "Type of Service" (ToS) possui 8 bits divididos em:

- 3 bits para prioridade (Precedence),
- 1 bit para minimizar atraso,
- 1 bit para maximizar taxa de transferência,
- 1 bit para maximizar confiabilidade,
- 1 reservado para uso futuro,
- 1 não utilizado.

O objetivo é fornecer diretrizes para roteadores priorizarem pacotes segundo o tipo de serviço desejado.

\textbf{Fonte:} J. Postel. \textit{RFC 791 – Internet Protocol}, setembro de 1981. Disponível em: \url{https://www.rfc-editor.org/rfc/rfc791}

\section*{Questão 2}

Temos 2048 bytes de dados + 20 bytes de cabeçalho TCP = 2068 bytes a serem entregues ao IP.

\vspace{0.5em}
\textbf{Primeira rede:}
\begin{itemize}
    \item MTU = 1024 bytes
    \item Cabeçalho IP = 20 bytes → carga útil = 1004 bytes
\end{itemize}

Fragmentos gerados:
\begin{itemize}
    \item Fragmento 1: offset 0, 1004 bytes de dados (bytes 0–1003)
    \item Fragmento 2: offset 1004, 1004 bytes (bytes 1004–2007)
    \item Fragmento 3: offset 2008, 60 bytes (bytes 2008–2067)
\end{itemize}

\vspace{0.5em}
\textbf{Segunda rede:}
\begin{itemize}
    \item MTU = 512 bytes
    \item Cabeçalho IP = 20 bytes → carga útil = 492 bytes
\end{itemize}

Fragmentos gerados:
\begin{itemize}
    \item Fragmento 1: offset 0, 492 bytes
    \item Fragmento 2: offset 492, 492 bytes
    \item Fragmento 3: offset 984, 492 bytes
    \item Fragmento 4: offset 1476, 492 bytes
    \item Fragmento 5: offset 1968, 100 bytes
\end{itemize}

Offsets IP são dados em unidades de 8 bytes:

\begin{itemize}
    \item Fragmento 1: offset 0  
    \item Fragmento 2: offset $\approx$ 61 (492 / 8)  
    \item Fragmento 3: offset 123  
    \item Fragmento 4: offset 185  
    \item Fragmento 5: offset 246
\end{itemize}

\textbf{Fonte:} KUROSE, J. F.; ROSS, K. W. \textit{Redes de Computadores e a Internet: Uma Abordagem Top-Down}. 7. ed. Pearson, 2018. Seção 3.2.2.

\section*{Questão 3}

O campo "Ident" tem 16 bits, ou seja, 65.536 valores possíveis. Para evitar colisões no intervalo de 60 segundos:

\[
\frac{65536~\text{pacotes}}{60~\text{s}} \approx 1092.27~\text{pacotes/s}
\]

Multiplicando pelo tamanho dos pacotes (576 bytes):

\[
1092.27 \times 576 = 628851.2~\text{bytes/s} \approx 5.03~\text{Mbps}
\]

\textbf{Conclusão:} a largura de banda máxima é cerca de 5 Mbps. Caso ultrapassada, pacotes com mesmo identificador podem ainda estar trafegando, o que causaria reassemblagem incorreta e perda de dados.

\textbf{Fonte:} Cálculo próprio baseado em especificações do IPv4 (RFC 791).

\section*{Questão 4}

A motivação principal do IPv6 foi o esgotamento iminente de endereços IPv4, limitados a 32 bits. O crescimento de dispositivos conectados exigiu uma solução escalável.

A demora na adoção do IPv6 se deve à complexidade da transição, ao custo de atualização de sistemas legados e à ampla adoção de mecanismos paliativos, como NAT, que estenderam a vida útil do IPv4.

\textbf{Fonte:} KUROSE, J. F.; ROSS, K. W. \textit{Redes de Computadores e a Internet: Uma Abordagem Top-Down}. 7. ed. Pearson, 2018. Seção 4.1.3.

\section*{Questão 5}

Principais inovações do IPv6 em relação ao IPv4:

\begin{itemize}
    \item Espaço de endereços maior (128 bits)
    \item Cabeçalho simplificado
    \item Suporte nativo a IPsec
    \item Autoconfiguração (stateless)
    \item Eliminação da necessidade de NAT
    \item Melhor suporte à mobilidade e QoS
\end{itemize}

\textbf{Fonte:} DEERING, S.; HINDEN, R. \textit{RFC 8200 – Internet Protocol, Version 6 (IPv6) Specification}, julho de 2017. \url{https://www.rfc-editor.org/rfc/rfc8200}

\section*{Questão 6}

\textbf{Pilha dupla (dual stack):} Permite que dispositivos operem com IPv4 e IPv6 simultaneamente, selecionando o protocolo com base no destino.

\textbf{Tunelamento (tunneling):} Envia pacotes IPv6 encapsulados em pacotes IPv4, permitindo que trafeguem por redes legadas sem suporte nativo a IPv6.

\textbf{Fonte:} KUROSE, J. F.; ROSS, K. W. \textit{Redes de Computadores e a Internet: Uma Abordagem Top-Down}. 7. ed. Pearson, 2018. Seção 4.1.3.

\end{document}